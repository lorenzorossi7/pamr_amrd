% Started from template downloaded from Wiley's 
% site, as per Manish's email:
%
% edbktmpl.tex. Current Version: June 5, 1999
%%%%%%%%%%%%%%%%%%%%%%%%%%%%%%%%%%%%%%%%%%%%%%%%%%%%%%%%%%%%%%%%
%
%  Template file for
%  Wiley edited book
%  Wiley Book Style, Design No.: SD 001E
%
%  Prepared by Amy Hendrickson, TeXnology Inc., March 1996.
%%%%%%%%%%%%%%%%%%%%%%%%%%%%%%%%%%%%%%%%%%%%%%%%%%%%%%%%%%%%%%%%

%%%%%%%%%%%%%%%%
% LaTeX2e:
\documentclass{w-edbk}

    % For TimesRoman Math (You must have MathTimes and MathTimes Plus 
    %                        font sets, order fonts from  www.yandy.com)
% \usepackage[mtbold,noTS1]{m-times}

    % For PostScript text, Computer Modern Math 
\usepackage{w-edbkps}

%%%%%%%%%%%%%%%%
% LaTeX2.09:
% \documentstyle{w-edbk}
    % For PostScript text, Computer Modern Math
% \documentstyle[w-edbkps]{w-edbk} 
    % For MathTimes and PostScript:
    %(m-times only works with LaTeX2e)

%%%%%%%%%%%%%%%%%%%%%%%%%%%%%%
%% Change options here if you want:
%%
%% How many levels of section head would you like numbered?
%% 0= no section numbers, 1= section, 2= subsection, 3= subsubsection
%%==>>
\setcounter{secnumdepth}{3}

%% How many levels of section head would you like to appear in the
%% Table of Contents?
%% 0= chapter titles, 1= section titles, 2= subsection titles, 
%% 3= subsubsection titles.
%%==>>
\setcounter{tocdepth}{1}
%%%%%%%%%%%%%%%%%%%%%%%%%%%%%%
%
% DRAFT
%
% Uncomment to get double spacing between lines, current date and time
% printed at bottom of page.
% \draft
% (If you want to keep tables from becoming double spaced also uncomment
% this):
% \renewcommand{\arraystretch}{0.6}
%%%%%%%%%%%%%%%%%%%%%%%%%%%%%%


\begin{document}

%% To be entered at Wiley, for final production: title page information,
%% table of contents, preface and introduction, and Part titles, and index. 
%% See edbksamp.tex for examples of how to enter the commands.

% \footnote{text} will cause a footnote to appear at the bottom of the page.

%% 
\title[]{PAMR: A "Second Generation" Infrastructure for Parallel Block Structured Finite Difference and Finite Volume Calculations}

%% Please supply author names in upper and lower case within square
%% brackets and in uppercase in curly brackets, i.e.,
%% \author[The Author]{THE AUTHOR\footnote{Presently on leave at
%% NASA, Houston, Texas, USA.}}
%% You may use a footnote for additional information.

\author[Matthew W. Choptuik]{MATTHEW W. CHOPTUIK}

%\affil{AT\& T Bell Laboratories\\
%Murray Hill, New Jersey}

\affil{University of British Columbia, Vancouver}

\author[Frans Pretorius]{FRANS PRETORIUS}
\affil{University of Alberta, Edmonton}

% prologue is optional. First arg is for text, second for author attribution.
% May be indexed and/or referenced. i.e.,

%\prologue{The sheer volumne of answers can often stifle insight...The purpose
%of computing\inxx{computing,the purpose} is insight, not numbers.}
%{Hamming}

%\prologue{}{}

%% Article text:

\section{Introduction}

body of article...

\section{Summary}

\begin{acknowledgments}

\end{acknowledgments}

%%% References should come before appendices unless there is
% a reference cite in the appendix, in which case references
% should come after the appendices.

% Necessary! Either
\begin{references}
\bibitem{mgamr} F. Pretorius and M.~W. Choptuik,
 ``Adaptive mesh refinement for coupled elliptic-hyperbolic systems'',
 {\em J. Comput. Phys.} {\bf 218} 246-274 (2006)
\end{references}

% or if you are using BibTeX:
%\chapbblname{thisfilename}
%\chapbibliography{your .bib file name}

%% appendix optional
%\appendix{This is the Appendix Title}
%This is an appendix with a title.

%\appendix{}
%This is an appendix without a title.

\end{document}

Other commands, and notes on usage:

-----
Possible section head levels:
\section{Introduction}
\subsection{This is subsection}
\subsubsection{This is subsubsection}
\paragraph{This is the paragraph}

-----
Captions:
 If you use index commands within a caption, precede \inx or \inxx with
 \protect.

\begin{figure}[h]
\caption{\protect\inx{Oscillograph} for memory address access ....
memory plane.}
\end{figure}

-----
Tables:
 Remember to use \centering for a small table and to start the table
 with \hline, use \hline underneath the column headers and at the end of 
 the table, i.e.,

\begin{table}[h]
\caption{Small Table}
\centering
\begin{tabular}{ccc}
\hline
one&two&three\\
\hline
C&D&E\\
\hline
\end{tabular}
\end{table}

For a table that expands to the width of the page, write

\begin{table}
\begin{tabular*}{\textwidth}{@{\extracolsep{\fill}}lcc}
\hline
....
\end{tabular*}
%% Sample table notes:
\begin{tablenotes}
$^a$Refs.~19 and 20.

$^b\kappa, \lambda>1$.
\end{tablenotes}
\end{table}

-----
Index commands:
\inx{term} will print `term' in text but will also send `term' and its
page number to the .inx file.

\inxx{term} will not print in text but will send term and its page number to
the .inx file.

\inxx{term,second term} will not print in text but will send `second term'
to inx file to print underneath `term' in the index.

1) Run Latex on file,
2) Run sort routine on file ie. `sort < filename.inx > filename.srt' on
your system to produce a filename.srt file
3)\printindex at end of book will input filename.srt and print index.

See documentation for more information.

-----
Algorithm.
Maintains same fonts as text (as opposed to verbatim which uses fixed
width fonts). Space at beginning of line will be maintained if you
use \ at beginning of line.

\begin{algorithm}
{\bf state\_transition algorithm} $\{$
\        for each neuron $j\in\{0,1,\ldots,M-1\}$
\        $\{$   
\            calculate the weighted sum $S_j$ using Eq. (6);
\            if ($S_j>t_j$)
\                    $\{$turn ON neuron; $Y_1=+1\}$   
\            else if ($S_j<t_j$)
\                    $\{$turn OFF neuron; $Y_1=-1\}$   
\            else
\                    $\{$no change in neuron state; $y_j$ remains %
unchanged;$\}$ .
\        $\}$   
$\}$   
\end{algorithm}

-----
Sample quote:
\begin{quote}
quotation...
\end{quote}

-----
Listing samples

\begin{enumerate}
\item
This is the first item in the numbered list.

\item
This is the second item in the numbered list.
\end{enumerate}

\begin{itemize}
\item
This is the first item in the itemized list.

\item
This is the first item in the itemized list.
This is the first item in the itemized list.
This is the first item in the itemized list.
\end{itemize}

\begin{itemize}
\item[]
This is the first item in the itemized list.

\item[]
This is the first item in the itemized list.
This is the first item in the itemized list.
This is the first item in the itemized list.
\end{itemize}

----
Glossary:

\begin{glossary}
\term{xxx}Text...
\term{yyy}Text...
\end{glossary}

i.e.,
\begin{glossary}
\term{GaAs}Gallium Arsinide. For similar device sizes GaAs transistors 
have three to
five times greater transconductance than those of of silicon bipolar
and MOS transistors.

\term{VLSI}Very Large Scale Integration. Since the mid-1970's 
VLSI technology has been successfully used in many areas, but its effect on
computers of all shapes and sizes has been the most dramatic. Some of the
application areas got boosts in performance while others became
feasible.

\end{glossary}



